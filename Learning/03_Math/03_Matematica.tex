\documentclass{article}
\begin{document}
\section{Math}
\begin{equation}
    x = 5
\end{equation}
\begin{equation}
    x^1 = 5
\end{equation}
\begin{equation}
\label{sommato}
    x_j = \sum^{0}_{i=n}m
\end{equation}
\begin{equation}
    x_j = \sum^{0}_{i=n} \frac{1}{2}m
\end{equation}
Le operazioni possono essere riferite sempre tramite il \textbf{lable}\\
Come nell'equazione \ref{sommato}

\section{Operazioni in Testo}
Per utilizzare le formule matematiche inserendole nel testo bisogna utilizzare l'environment math in line scrivendo il doppio singolo dollaro $ x_j = \sum^{0}_{i=n}m$ questa è una sommatoria 


\end{document}
