\documentclass{article}

\usepackage{lipsum}


\begin{document}
\tableofcontents
\section{Come si crea l'indice}
Per creare l'indice è sifficiente usare il tag \textbf{$\backslash$tableofcontents} prima di tutte le sezioni che compongono il documento
\section{Bibliografia}
Si Inserisce nel file .bib del progetto il libro/documento etc... e poi lo citiamo tramite il tag \textbf{cite} nel testo in cui compare. Dopo con il tag \textbf{bibliography} si crea la bibliografia dove vogliamo \cite{adams1995hitchhiker}. Nel testo ci saranno gli indici associati che nel pdf saranno cliccabili. 
%%%%%%%%%%%%%%%%%%%%%%%%%%%%%%%%%%%%%%%%%%%%%%%%%%%%%%%%%%%%%%%%
\section{Prima}
\lipsum[1]  %stampa testo d'esempio 
\subsection{PrimaAnnidata}
\lipsum[1]

%%%%%%%%%%%%%%%%%%%%%%%%%%%%%%%%%%%%%%%%%%%%%%%%%%%%%%%%%%%%%%%%
\section{Seconda}
\lipsum[1]

%%%%%%%%%%%%%%%%%%%%%%%%%%%%%%%%%%%%%%%%%%%%%%%%%%%%%%%%%%%%%%%%
\section{Terza}
\lipsum[1]
\subsection{TerzaAnnidata1}
\lipsum[1]
\subsection{TerzaAnnidata2}
\lipsum[1]
%%%%%%%%%%%%%%%%%%%%%%%%%%%%%%%%%%%%%%%%%%%%%%%%%%%%%%%%%%%%%%%%

\bibliographystyle{plain}
\bibliography{references.bib}
\end{document}