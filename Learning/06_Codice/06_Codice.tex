\documentclass{article}
\usepackage[utf8]{inputenc}

\usepackage{listings}
\usepackage{xcolor}

%=== CODICE  =========================
% Definiamo i colori per rendere il codice più leggibile
\usepackage{xcolor} 
\definecolor{commentgreen}{RGB}{2,112,10}
\definecolor{eminence}{RGB}{108,48,130}
\definecolor{weborange}{RGB}{255,165,0}
\definecolor{frenchplum}{RGB}{129,20,83}

\usepackage{listings}
\lstset {
    language= C,
    frame=tb,
    tabsize=4,
    showstringspaces=false,
    numbers=left,
    %upquote=true,
    commentstyle=\color{commentgreen},
    keywordstyle=\color{eminence},
    stringstyle=\color{red},
    basicstyle=\small\ttfamily, % basic font setting
    emph={int,char,double,float,unsigned,void,bool},
    emphstyle={\color{blue}},
    % keyword highlighting
    classoffset=1, % starting new class
    otherkeywords={>,<,.,;,-,!,=,~},
    morekeywords={>,<,.,;,-,!,=,~},
    keywordstyle=\color{weborange},
    classoffset=0,
}


\title{Code Listing}
\date{ 03.03.2022}

\begin{document}

\maketitle  


\section{C Code examples}

\begin{lstlisting}[caption={level2.c},label={lst:lst2},language=C]
#include <stdlib.h>
#include <unistd.h>
#include <string.h>
#include <sys/types.h>
#include <stdio.h>
int main(int argc, char **argv, char **envp){
char *buffer;
gid_t gid;
uid_t uid;
gid = getegid();
uid = geteuid();
setresgid(gid, gid, gid);
setresuid(uid, uid, uid);

buffer = NULL;

asprintf(&buffer, "/bin/echo %s is cool", getenv("USER"));
printf("about to call system(\"%s\")\n", buffer);
system(buffer);
}
\end{lstlisting}

%% inseriamo il codice in una nuova pagina
\newpage 

\section{Java Code examples}

\begin{lstlisting}[language=Java, caption=Java example]
// Hello.java
import javax.swing.JApplet;
import java.awt.Graphics;

public class Hello extends JApplet {
    public void paintComponent(Graphics g) {
        g.drawString("Hello, world!", 65, 95);
    }    
}
\end{lstlisting}


%% inseriamo il codice in una nuova pagina
\newpage 

\section{Python Code examples}

\begin{lstlisting}[language=Python, caption=Python example]
import numpy as np
    
def incmatrix(genl1,genl2):
    m = len(genl1)
    n = len(genl2)
    M = None #to become the incidence matrix
    VT = np.zeros((n*m,1), int)  #dummy variable
    
    #compute the bitwise xor matrix
    M1 = bitxormatrix(genl1)
    M2 = np.triu(bitxormatrix(genl2),1) 

    for i in range(m-1):
        for j in range(i+1, m):
            [r,c] = np.where(M2 == M1[i,j])
            for k in range(len(r)):
                VT[(i)*n + r[k]] = 1;
                VT[(i)*n + c[k]] = 1;
                VT[(j)*n + r[k]] = 1;
                VT[(j)*n + c[k]] = 1;
                
                if M is None:
                    M = np.copy(VT)
                else:
                    M = np.concatenate((M, VT), 1)
                
                VT = np.zeros((n*m,1), int)
    
    return M
\end{lstlisting}


%% INSERIAMO L'INDICE DI TUTTI I CODICI INSERITI NEL PROGETTO
\newpage
\lstlistoflistings

\end{document}
