\chapter{Convenzioni Git}
\label{chap:convenzioni}


\section{Best practice per il branching}
    \begin{itemize}
        \item Lavorare sempre su branch propri
        \item Staccare i branch da staging (versione stabile della repository)
        \item Nominare il branch con la specifica della funzionalità da implementare (Es. branch: update-users-api)
        \item Aggiornare il proprio branch prima di ogni merge request (attraverso il pull da staging)
        \item Assegnare la merge request al scrum master di riferimento (Eg. Backend: Assignee -> Alessandro)
    \end{itemize}

\section{Best practice per le commit}
    \subsection{Creazione delle commit}
        \begin{itemize}
            \item Aggiungere nella commit solo i cambiamenti relativi ad una singola funzionalità. Evitare quindi di creare delle commit di grosse dimensioni, contenenti modifiche di diverso genere. 
            \item Testare accuratamente le funzionalità interessate prima di ogni commit.
            \item Aggiornare il file .gitignore in caso di aggiunta file spuri.
        \end{itemize}
    
    \subsection{Scrittura delle commit}
        \begin{itemize}
            \item Specificare il tipo di operazione effettuata
            \item Specificare l’identificativo della issue risolta
            \item Specificare, con un breve messaggio [max 50 caratteri] in lingua inglese, l’operazione che è stata effettuata nella commit 
        
        \end{itemize}

\subsubsection{Template}

\begin{large}
\centering{ git commit -m “\{Method\} \#\{id-issue\} – \{Commit message\}”}
\end{large}

\subsubsection{Esempi}
\textbf{Implementazione di una nuova funzionalità:}
\begin{center}
    git commit -m “Implementation \#23 – Add get all users API”
\end{center}
\medskip \medskip
%=================================================================================
\textbf{Aggiornamento di una funzionalità implementata in precedenza: }
\begin{center}
    git commit -m “Update \#11 – Remove unused fields to UserDto class”
\end{center}
\medskip \medskip
%=================================================================================
\textbf{Risoluzione di un bug o di una issue: }
\begin{center}
    git commit -m “Fix  \#103 – Fix delete user API issue”
\end{center}
\medskip \medskip
%=================================================================================
Qualora dovesse risultare necessaria una commit non associata a nessuna Issue, procedere semplicemente con il messaggio della commit.\\
\textbf{Esempi: }
\begin{center}
    \begin{itemize}
        \item git commit -m “Code refactoring” 
        \item git commit -m “Remove unused DTOs”
    \end{itemize}
\end{center}
