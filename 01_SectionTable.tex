\documentclass{article}
\begin{document}

%Sto scrivendo un commento

Bisogna specificare qual è il file main.tex da cui parte la compilazione
Si specifixa il tipo di documento tra le parentesi
Ogni tag si dichiara con uno slash 
Ogni documento ha un tag di inizio ed uno di fine che delimitano il documento

\section{Prima Sezione - A capo}
\label{sezioneACapo}
Lorem Ipsum is simply dummy text of the printing and typesetting industry.

Se vado a capo semplicemente si ha un piccolo spazio sulla sinistra. 

\noindent Vado a capo senza lasciare lo sapzio sulla sinistra con il tag noindent. 

\subsection{Sezione Interna - Caratteri}
\textit{Font Italico}
\textbf{Grassetto ctrl+B}
\underline{Sottolineato}

\subsubsection{Sezione Interna 2 - Liste Puntate}
Si inizia un nuovo blocco con i tag begin e tra le parentesi la classe \textbf{itemize} per avere i punti, altrimenti si utilizza come classe \textbf{enumerate} per avere i numeri
\begin{itemize}
    \item elemento puntato 1
    \item elemento puntato 2
    \item elemento puntato 3
\end{itemize}
\begin{enumerate}
    \item elemento numerico 1
    \item elemento numerico 2
    \item elemento numerico 3
\end{enumerate}
Per creare una lista di lista si annidano i tag \textbf{begin}
\begin{enumerate}
\item elemento padre 1
\item elemento padre 2
    \begin{itemize}
        \item elemento figlio 1
        \item elemento figlio 2
        \item elemento figlio 3
    \end{itemize}
\end{enumerate}
Per specificare il simbolo dell'elenco si usano le []
\begin{itemize}
    \item [*] stella
\end{itemize}

\paragraph{Paragrafo} - Sezioni e subsezioni 

\section{Tabelle}
Per inserire una tabella è sufficiente scrivere il tag Begin e come classe \textbf{table}

 il tag \textbf{hline} identifica le righe orizzontali ed i campi si scrivono con and\textbf{e commerciale}
    slesh slesh va a capo tra le varie colonne. [0.5ex] è la distanza tra gli elementi
\begin{table}[h!]
%centering centra la tabella
    \centering
    % Nella "zona table c'è una zona "tabular" che contiene la tabella"
    % Il simbolo | tra le c (centred) oppure l (left) oppure r (rigth) indica le righe delle colonne e dove % si scriverà il testo all'interno 
    
    \begin{tabular}{c|c|l}
    \hline
    Col1 & Col2 & Col3 \\[0.5ex]
    \hline\hline
    1   &   2   &   3  \\
    4   &   5   &   6 \\ [5ex]
    \hline %disegna l'ultima riga 
    \end{tabular}
    \caption{Prova Tabella 1}
    \label{TabellaUno}
\end{table}

\begin{table}[h!]
    \centering
    \begin{tabular}{c|c|l}
    \hline
    Col1 & Col2 & Col3 \\[0.5ex]
    \hline\hline
    10   &   20   &   30  \\
    40   &   50   &   60 \\ [5ex]
    \hline 
    \end{tabular}
    \caption{Prova Tabella 2}
    \label{TabellaDue}
\end{table}

\subsection{Reference}
come illustrato in tabella \ref{TabellaUno} Usiamo il tag \textbf{ref} per puntare alla tabella che ha come tag \textbf{label} quello inserito tra le graffe {}. \\
In questo modo le dipendenze si aggiorneranno automaticamente in caso di aggiornamento.\\
Si possono riferite anche sezioni se si inserisce il tag \textbf{label}. \\
Sezione Riferita \ref{sezioneACapo}
\end{document}
